\documentclass{article}

\usepackage{polski}
\usepackage[utf8]{inputenc}
\usepackage{kpfonts}
\usepackage{color}

\newcommand*\Hs[1]{\ensuremath{\color{red}\varheartsuit}}
\newcommand*\Ss[1]{\ensuremath{\color{black}\spadesuit}}
\newcommand*\Ds[1]{\ensuremath{\color{red}\vardiamondsuit}}
\newcommand*\Cs[1]{\ensuremath{\color{black}\clubsuit}}
\newcommand*\NT[1]{{\color{black}\textsc{NT}}}

\begin{document}
    \section{Rebid otwierającego}
    Ręce otwierającego dzielimy na 3 strefy siły:
    	\begin{itemize}
    	\item przedział dolny - 13-16 pkt.
    	\item przedział średni - 17-18 pkt.
    	\item przedział górny - 19+ pkt.
    	\end{itemize}
    	\subsection{Rebid otwierającego po podniesieniu starszego koloru otwarcia na poziomie 2}
    	\begin{itemize}
    		\item 13-16 pkt.
    			\begin{itemize}
    				\item pas
    			\end{itemize}
    		\item 17-18 pkt.
    			\begin{itemize}
    				\item podniesienie swojego koloru na poziom 3
    				\item zalicytowanie 2NT
    				\item zgłoszenie koloru bocznego
    			\end{itemize}
    		\item 19+ pkt.
    			\begin{itemize}
    				\item końcówka w uzgodnionym kolorze
    			\end{itemize}
    	\end{itemize}
    	\subsection{Rebid otwierającego po podniesieniu młodszego koloru otwarcia na poziomie 2}
    	\begin{itemize}
    		\item 13-16 pkt.
    			\begin{itemize}
    				\item pas
    			\end{itemize}
    		\item 17-18 pkt.
    			\begin{itemize}
    				\item podniesienie swojego koloru na poziom 3
    				\item zalicytowanie 2NT
    				\item zgłoszenie koloru bocznego
    			\end{itemize}
    		\item 19+ pkt.
    			\begin{itemize}
    				\item 3NT
    				\item końcówka w kolor uzgodniony
    				\item zgłoszenie koloru bocznego
    			\end{itemize}
    	\end{itemize}
    	\subsection{Rebid otwierającego po podniesieniu koloru otwarcia na poziomie 3}
    	\begin{itemize}
    		\item 13-14 pkt.
    			\begin{itemize}
    				\item pas
    			\end{itemize}
    		\item 15+ pkt.
    			\begin{itemize}
    				\item końcówka w uzgodniony kolor starszy (gdy mamy uzgodniony kolor starszy)
    				\item 3NT (gdy mamy uzgodniony kolor młodszy)
    				\item zgłoszenie koloru bocznego (gdy mamy uzgodniony kolor młodszy)
    				\item końcówka w uzgodniony kolor młodszy (gdy mamy uzgodniony kolor młodszy)
    			\end{itemize}
    	\end{itemize}
    	\subsection{Rebid otwierającego po odpowiedzi 1NT}
    	\begin{itemize}
    		\item 13-16 pkt.
    			\begin{itemize}
    				\item pas
    				\item swój drugi kolor o ile można go zgłosić poniżej koloru otwarcia
    				\item powtarzamy kolor otwarcia jeżeli mamy w nim 6+ kart
    			\end{itemize}
    		\item 17-18 pkt.
    			\begin{itemize}
    				\item 2NT z ręką zrównoważoną
    				\item boczny kolor ekonomicznie lub rewersem
    				\item Kolor otwarcia 6+ ze skokiem
    			\end{itemize}
    		\item 19+ pkt.
    			\begin{itemize}
    				\item 3NT z ręką zrównoważoną
    				\item boczny kolor ze skokiem lub rewersem
    				\item końcówka w kolor otwarcia (z solidnym kolorem otwarcia 6+ - tylko w stary kolor)
    			\end{itemize}
    	\end{itemize}
    	\subsection{Rebid otwierającego po odpowiedzi nowym kolorem na poziomie 1}
    	\begin{itemize}
    		\item 13-16 pkt.
    			\begin{itemize}
    				\item z fitem czterokartowym podnieść kolor odpowiedzi na poziom 2
    				\item nowy kolor własny zgłoszony ekonomicznie
    				\item 1NT z ręką zrównoważoną
    				\item powtarzamy kolor otwarcia jeżeli mamy w nim 6+ kart
    			\end{itemize}
    		\item 17-18 pkt.
    			\begin{itemize}
    				\item z fitem czterokartowym podnieść kolor odpowiedzi na poziom 3
    				\item nowy kolor zgłoszony ekonomicznie lub rewersem
    				\item powtarzamy ze skokiem kolor otwarcia jeżeli mamy w nim 6+ kart
    				\item 2NT z ręką zrównoważoną
    			\end{itemize}
    		\item 19+ pkt.
    			\begin{itemize}
    				\item z fitem czterokartowym podnieść kolor odpowiedzi na poziom 4
    				\item nowy kolor zgłoszony ze skokiem lub rewersem
    				\item 3NT
    			\end{itemize}
    	\end{itemize}
    	\subsection{Rebid otwierającego po odpowiedzi nowym kolorem na poziomie 2}
    	\begin{itemize}
    		\item 13-15 pkt.
    			\begin{itemize}
    				\item .
    			\end{itemize}
    		\item 16+ pkt.
    			\begin{itemize}
    				\item .
    			\end{itemize}
    	\end{itemize}
\end{document}
