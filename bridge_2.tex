\documentclass{article}

\usepackage{polski}
\usepackage[utf8]{inputenc}
\usepackage{kpfonts}
\usepackage{color}

\newcommand*\Hs[1]{\ensuremath{\color{red}\varheartsuit}}
\newcommand*\Ss[1]{\ensuremath{\color{black}\spadesuit}}
\newcommand*\Ds[1]{\ensuremath{\color{red}\vardiamondsuit}}
\newcommand*\Cs[1]{\ensuremath{\color{black}\clubsuit}}
\newcommand*\NT[1]{{\color{black}\textsc{NT}}}

\begin{document}
    \section{Otwarcia na poziomie 2}
    	\begin{itemize}
    	\item 2 \Cs{} - 24+ pkt. (PC+PD), skład dowolny
    	\item 2 \Ds{} - 6-11 pkt. (PC), nienajgorszy sześciokart \Ds{}
    	\item 2 \Hs{} - 6-11 pkt. (PC), nienajgorszy sześciokart \Hs{}
    	\item 2 \Ss{} - 6-11 pkt. (PC), nienajgorszy sześciokart \Ss{}
    	\item 2 \NT{} - 20-22 pkt. (PC+PD), skład zrównoważony (brak krótkości, brak 2 dubli), brak 5+ \Hs{} / \Ss{}
    	\end{itemize}
    \section{Skład zrównoważony}
    	Składem zrównoważonym nazywamy dowolny skład w którym nie ma krótkości (singli i renonsów) oraz jest maksymalnie jeden dubel:
    	\begin{itemize}
    	\item 4432
    	\item 5332
    	\item 4333
    	\end{itemize}
    	Można rozróżnić jeszcze ręce semi-zrównoważone:
    	\begin{itemize}
    	\item 6322
    	\item 5422
    	\end{itemize}
    \section{Priorytety w licytacji naturalnej}
    \begin{enumerate}
    	\item Uzgodnienie koloru starszego
    	\item Zgłoszenie swojego koloru starszego
    	\item Uzgodnienie koloru młodszego
    	\item Zgłoszenie odzywki NT na odpowiednim poziomie (nie zgłaszaj koloru młodszego, jeżeli masz w nim tylko 4 karty i skład zrównoważony)
    \end{enumerate}
    \section{Odpowiedzi po otwarciu 1 w kolor}
		\begin{enumerate}
  			\item Siła 0-5 pkt.
  			\begin{enumerate}
   				 \item pas
  			\end{enumerate}
  			\item Siła 6-10 pkt.
  			\begin{enumerate}
   				 \item Uzgodnienie koloru partnera na wysokości 2
   				 \item Zgłoszenie nowego koloru na wysokości 1 (4+ karty w nowym kolorze)
   				 \item Zgłoszenie 1 NT
  			\end{enumerate}
  			\item Siła 11-12 pkt.
  			\begin{enumerate}
   				 \item Uzgodnienie koloru partnera na wysokości 3
   				 \item Zgłoszenie nowego koloru na wysokości 1 lub 2 (4+ karty w nowym kolorze)
   				 \item Zgłoszenie 2 NT
  			\end{enumerate}
  			\item Siła 13+ pkt.
  			\begin{enumerate}
   				 \item Skok nowym kolorem (5+ niezłych kart w nowym kolorze)
   				 \item Zgłoszenie nowego koloru na wysokości 1 lub 2 (4+ karty w nowym kolorze)
   				 \item Podniesienie koloru partnera do końcówki
   				 \item Skok na 3NT
  			\end{enumerate}
		\end{enumerate}
\end{document}
